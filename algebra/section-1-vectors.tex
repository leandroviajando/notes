\section{Vectors}

\begin{paracol}{2}

\begin{tikzpicture}
\node [rounded-box] (box){\begin{minipage}{0.45\textwidth}
    $\mathbb{R}^2$ is the set of vectors in the plane.

    $\mathbb{R}^3$ is the set of vectors in space.

    $\mathbb{R}^n = \begin{bmatrix} a_1 \\ \vdots \\ a_n \end{bmatrix}$ with $a_1, \dots, a_n \in \mathbb{R}$ is the set of vectors with $n$ components. \\

    \textbf{Binary vectors} and modular arithmetic: The set of all $m$-ary vectors of length $n$ is denoted by $\mathbb{Z}_m^n$.
\end{minipage}};
\node[rounded-box-title, left=10pt] at (box.north east) {Definition};
\end{tikzpicture}

\switchcolumn

\begin{tikzpicture}
\node [rounded-box] (box){\begin{minipage}{0.45\textwidth}
    \textbf{Velocity, speed and acceleration} in $\mathbb{R}^2$: \\

    \begin{itemize}
        \item Displacement at time $t$: $\mathbf{s}(t) = \begin{pmatrix}
            x(t) \\ y(t)
        \end{pmatrix}$
        \item Velocity at time $t$: $\mathbf{v}(t) = \begin{pmatrix}
            x'(t) \\ y'(t)
        \end{pmatrix}$
        \item Speed at time $t$: $|| \mathbf{v}(t) || = \sqrt{(x'(t))^2 + (y'(t))^2}$
        \item Acceleration at time $t$: $\mathbf{a}(t) = \mathbf{v}'(t) = \begin{pmatrix}
            x''(t) \\ y''(t)
        \end{pmatrix}$
    \end{itemize}
\end{minipage}};
\node[rounded-box-title, left=10pt] at (box.north east) {Definition};
\end{tikzpicture}

\end{paracol}

\subsection{The Dot Product}

\begin{paracol}{2}

\begin{tikzpicture}
\node [rounded-box] (box){\begin{minipage}{0.45\textwidth}
    The \textbf{dot product} of vectors $\mathbf{u}, \mathbf{v}$ is a similarity measure:

    \vspace{-10pt}

    \begin{align}
        \mathbf{u} \cdot \mathbf{v} & = u_1 v_1 + \dots + u_n v_n \\
            & = || \mathbf{u} || || \mathbf{v} || \cos(\theta)
    \end{align}

    $\cos(\theta)$ is largest when the two vectors point in the same direction, because $\cos(0) = 1$.
\end{minipage}};
\node[rounded-box-title, left=10pt] at (box.north east) {Definition};
\end{tikzpicture}

\begin{tikzpicture}
\node [rounded-box] (box){\begin{minipage}{0.45\textwidth}
    Suppose $\mathbf{u}, \mathbf{v} \neq \mathbf{0}$. Then

    \vspace{5pt}

    \begin{enumerate}
        \item $\mathbf{u} \cdot \mathbf{v} > 0 \iff \theta \in [0, \frac{\pi}{2}]$ \quad (acute angle)
        \item $\mathbf{u} \cdot \mathbf{v} = 0 \iff \theta = \frac{\pi}{2}$ \hspace{11pt} \quad (\textbf{orthogonal})
        \item $\mathbf{u} \cdot \mathbf{v} < 0 \iff \theta \in [\frac{\pi}{2}, \pi]$ \quad (obtuse angle)
        \item The zero vector is orthogonal to all vectors: $\mathbf{0} \cdot \mathbf{u} = 0$
        \item $\mathbf{u} \cdot \mathbf{u} = || \mathbf{u} || || \mathbf{u} || \cos{0} =  || \mathbf{u} ||^2$
    \end{enumerate}
\end{minipage}};
\node[rounded-box-title, left=10pt] at (box.north east) {Theorem};
\end{tikzpicture}

\begin{tikzpicture}
\node [rounded-box] (box){\begin{minipage}{0.45\textwidth}
    \textbf{The angle formula}: For non-zero vectors $\mathbf{u}$ and $\mathbf{v}$:

    $$\cos(\theta) = \frac{\mathbf{u} \cdot \mathbf{v}}{|| \mathbf{u} || || \mathbf{v} ||}$$
\end{minipage}};
\node[rounded-box-title, left=10pt] at (box.north east) {Theorem};
\end{tikzpicture}

\switchcolumn

\begin{tikzpicture}
\node [rounded-box] (box){\begin{minipage}{0.45\textwidth}
    \textbf{Rules of calculation}: Let $\mathbf{u}, \mathbf{v}, \mathbf{w} \in \mathbb{R}^n, c \in \mathbb{R}$. Then \\

    \begin{enumerate}
        \item \makebox[4.5cm][l]{$\mathbf{u} \cdot \mathbf{v} = \mathbf{v} \cdot \mathbf{u}$} \quad (commutativity)
        \item \makebox[4.5cm][l]{$\mathbf{u} \cdot (\mathbf{v} + \mathbf{w}) = \mathbf{u} \cdot \mathbf{v} + \mathbf{u} \cdot \mathbf{w}$} \quad (distributivity)
        \item $(c \mathbf{u}) \cdot \mathbf{v} = \mathbf{u} \cdot (c \mathbf{v}) = c (\mathbf{u} \cdot \mathbf{v})$
        \item $\mathbf{u} \cdot \mathbf{u} = || \mathbf{u} ||^2 \geq 0$
    \end{enumerate}
\end{minipage}};
\node[rounded-box-title, left=10pt] at (box.north east) {Theorem};
\end{tikzpicture}

\begin{tikzpicture}
\node [rounded-box] (box){\begin{minipage}{0.45\textwidth}
    The \textbf{norm} of a vector is given by:

    $$|| \mathbf{v} || = \sqrt{\mathbf{v} \cdot \mathbf{v}} \geq 0$$

    If the norm of a vector is equal to one, it is \textbf{normalised}. \\

    If a set of vectors are mutually orthogonal and normalised, they are \textbf{orthonormal}.
\end{minipage}};
\node[rounded-box-title, left=10pt] at (box.north east) {Definition};
\end{tikzpicture}

\begin{tikzpicture}
\node [rounded-box] (box){\begin{minipage}{0.45\textwidth}
    The \textbf{distance} between two vectors:

    \vspace{-5pt}

    $$d(\mathbf{u}, \mathbf{v}) = || \mathbf{u} - \mathbf{v} || = \sqrt{(u_1 - v_1)^2 + \dots + (u_n - v_n)^2}$$
\end{minipage}};
\node[rounded-box-title, left=10pt] at (box.north east) {Theorem};
\end{tikzpicture}

\end{paracol}

\begin{tikzpicture}
\node [rounded-box] (box){\begin{minipage}{0.975\textwidth}
    Let $P$ be a point on a circle with centre $C$. The line tangent to the circle at $P$ is orthogonal to the vector $\vec{CP}$.
\end{minipage}};
\node[rounded-box-title, left=10pt] at (box.north east) {Theorem};
\end{tikzpicture}

\begin{tikzpicture}
\node [rounded-box] (box){\begin{minipage}{0.975\textwidth}
    $$||\mathbf{v} - \mathbf{w}||^2 = ||\mathbf{v}||^2 + ||\mathbf{w}||^2 \text{ if and only if } \mathbf{v}, \mathbf{w} \text{ are orthogonal.}$$
\end{minipage}};
\node[rounded-box-title, left=10pt] at (box.north east) {The Pythagorean Theorem};
\end{tikzpicture}

\textbf{Proof}:

\vspace{-40pt}

\begin{align}
||\mathbf{v} - \mathbf{w}||^2 & = (\mathbf{v} - \mathbf{w}) \cdot (\mathbf{v} - \mathbf{w}) & \\
    & = \mathbf{v} \cdot \mathbf{v} + \mathbf{v} \cdot (- \mathbf{w}) + (- \mathbf{w}) \cdot \mathbf{v} + (- \mathbf{w}) \cdot (- \mathbf{w}) & \text{by rule 2} \\
    & = \mathbf{v} \cdot \mathbf{v} - 2 \mathbf{v} \cdot \mathbf{w} + \mathbf{w} \cdot \mathbf{w} & \text{by rules 1 and 2} \\
    & = ||\mathbf{v}||^2 - 2 \mathbf{v} \cdot \mathbf{w} + ||\mathbf{w}||^2 & \\
    & = ||\mathbf{v}||^2 + ||\mathbf{w}||^2 & \text{ if and only if }\mathbf{v} \cdot \mathbf{w} = 0
\end{align}

\begin{paracol}{2}

\begin{tikzpicture}
\node [rounded-box] (box){\begin{minipage}{0.45\textwidth}
    $$| \mathbf{u} \cdot \mathbf{v} | \leq || \mathbf{u} || || \mathbf{v} ||$$
\end{minipage}};
\node[rounded-box-title, left=10pt] at (box.north east) {Cauchy-Schwarz Inequality};
\end{tikzpicture}

\textbf{Proof}: TODO

\switchcolumn

\begin{tikzpicture}
\node [rounded-box] (box){\begin{minipage}{0.45\textwidth}
    $$| \mathbf{u} + \mathbf{v} | \leq || \mathbf{u} || + || \mathbf{v} ||$$
\end{minipage}};
\node[rounded-box-title, left=10pt] at (box.north east) {Triangle Inequality};
\end{tikzpicture}

\textbf{Proof}: TODO

\end{paracol}

\newpage

\subsection{The Cross Product}

\begin{paracol}{2}

\begin{tikzpicture}
\node [rounded-box] (box){\begin{minipage}{0.45\textwidth}
    For vectors $\mathbf{u}, \mathbf{v} \in \mathbb{R}^3$, the \textbf{cross product} of $\mathbf{u}$ and $\mathbf{v}$ is:

    $$\mathbf{v} \times \mathbf{v} = \begin{bmatrix}
        u_1 \\ u_2 \\ u_3
    \end{bmatrix} \times \begin{bmatrix}
        v_1 \\ v_2 \\ v_3
    \end{bmatrix} = \begin{bmatrix}
        u_2 v_3 - u_3 v_2 \\
        u_3 v_1 - u_1 v_3 \\
        u_1 v_2 - u_2 v_1
    \end{bmatrix}$$
\end{minipage}};
\node[rounded-box-title, left=10pt] at (box.north east) {Definition};
\end{tikzpicture}

\begin{tikzpicture}
\node [rounded-box] (box){\begin{minipage}{0.45\textwidth}
    \textbf{Rules of calculation}: \\

    \begin{enumerate}
        \item \textbf{The right-hand rule}: $\mathbf{v} \times \mathbf{u} = - (\mathbf{u} \times \mathbf{v})$
        \item It follows that $\mathbf{u} \times \mathbf{u} = \mathbf{0}$; and $\mathbf{u} \times \mathbf{0} = \mathbf{0}$. \\

        \item $\mathbf{u} \times k \mathbf{v} = k (\mathbf{u} \times \mathbf{v})$
        \item $\mathbf{u} \times k \mathbf{u} = \mathbf{0}$ \\

        \item $\mathbf{u} \times (\mathbf{v} + \mathbf{w}) = \mathbf{u} \times \mathbf{v} + \mathbf{u} \times \mathbf{w}$
        \item $\mathbf{u} \cdot (\mathbf{v} \times \mathbf{w}) = (\mathbf{u} \times \mathbf{v}) \cdot \mathbf{w}$
        \item $\mathbf{u} \times (\mathbf{v} \times \mathbf{w}) = (\mathbf{u} \cdot \mathbf{w}) \mathbf{v} - (\mathbf{u} \cdot \mathbf{w}) \mathbf{w}$ \\

        \item $|| \mathbf{u} \times \mathbf{v} ||^2 = || \mathbf{u} ||^2 || \mathbf{v} ||^2 - (\mathbf{u} \cdot \mathbf{v})^2$
    \end{enumerate}
\end{minipage}};
\node[rounded-box-title, left=10pt] at (box.north east) {Definition};
\end{tikzpicture}

\switchcolumn

\begin{tikzpicture}
\node [rounded-box] (box){\begin{minipage}{0.45\textwidth}
    The cross product of two vectors $\mathbf{u}$ and $\mathbf{v}$ in $\mathbb{R}^3$ is orthogonal to both $\mathbf{u}$ and $\mathbf{v}$: $(\mathbf{u} \times \mathbf{v}) \perp \mathbf{u}, \qquad (\mathbf{u} \times \mathbf{v}) \perp \mathbf{v}$
\end{minipage}};
\node[rounded-box-title, left=10pt] at (box.north east) {Theorem};
\end{tikzpicture}

\begin{tikzpicture}
\node [rounded-box] (box){\begin{minipage}{0.45\textwidth}
    Two non-zero vectors $\mathbf{u}$ and $\mathbf{v}$ are \textbf{parallel} iff $\mathbf{u} \times \mathbf{v} = \mathbf{0}$.
\end{minipage}};
\node[rounded-box-title, left=10pt] at (box.north east) {Theorem};
\end{tikzpicture}

\begin{tikzpicture}
\node [rounded-box] (box){\begin{minipage}{0.45\textwidth}
    Let $\mathbf{a}$ and $\mathbf{b}$ be vectors in $\mathbb{R}^3$. Then the parallelogram spanned by these vectors has area equal to the determinant of their cross product:

    $$\text{Area} = || \mathbf{a} \times \mathbf{b} || = || \mathbf{a} || || \mathbf{b} || \sin(\theta)$$
\end{minipage}};
\node[rounded-box-title, left=10pt] at (box.north east) {Theorem};
\end{tikzpicture}

\begin{tikzpicture}
\node [rounded-box] (box){\begin{minipage}{0.45\textwidth}
    The volume of a parallelepiped spanned by vectors $\mathbf{a}, \mathbf{b}$ and $\mathbf{c}$ in $\mathbb{R}^3$ is given by (the absolute value of) the vectors' \textbf{triple product}:

    $$\text{Volume} = | (\mathbf{a} \times \mathbf{b}) \cdot \mathbf{c} | = || \mathbf{a} \times \mathbf{b} || || \mathbf{c} || | \cos(\theta) |$$
\end{minipage}};
\node[rounded-box-title, left=10pt] at (box.north east) {Theorem};
\end{tikzpicture}

\end{paracol}

\vspace{-10pt}

\subsection{Lines and Planes}

\begin{paracol}{2}

\begin{tikzpicture}
\node [rounded-box] (box){\begin{minipage}{0.45\textwidth}
    In $\mathbb{R}^2$, $n_1 x + n_2 y = c$ describes a line orthogonal to $\begin{bmatrix}
        n_1 \\ n_2
    \end{bmatrix}$. \\

    If $n_2 \neq 0$, there exists a unique representation of the line:

    $$y = a x + b \text{ where } a = - \frac{n_1}{n_2} \text{ and } b = \frac{c}{n_2}$$

    In $\mathbb{R}^3$, $n_1 x + n_2 y + n_3 z = c$ describes a plane orthogonal to $\begin{bmatrix}
        n_1 & n_2 & n_3
    \end{bmatrix}^T$. \\

    In $\mathbb{R}^k$, $n_1 x + n_2 x_2 + \dots + n_k x_k = c$ describes a hyperplane orthogonal to $\begin{bmatrix}
        n_1 & n_2 & \dots & n_k
    \end{bmatrix}^T$. \\
\end{minipage}};
\node[rounded-box-title, left=10pt] at (box.north east) {Theorem};
\end{tikzpicture}

\switchcolumn

\begin{tikzpicture}
\node [rounded-box] (box){\begin{minipage}{0.45\textwidth}
    The distance to the centre of a circle is equal to its norm:

    $$
    r = || \begin{pmatrix}
        x \\ y
    \end{pmatrix} - \begin{pmatrix}
        a \\ b
    \end{pmatrix} || = \sqrt{(x-a)^2 + (y-b)^2}
    $$

    This gives the equation of a \textbf{circle}:

    \vspace{-10pt}

    $$(x-a)^2 + (y-b)^2 = r^2$$
\end{minipage}};
\node[rounded-box-title, left=10pt] at (box.north east) {Theorem};
\end{tikzpicture}

\begin{tikzpicture}
\node [rounded-box] (box){\begin{minipage}{0.45\textwidth}
    Similarly in $\mathbb{R}^3$, the distance to the centre of a sphere is:

    \vspace{-7.5pt}

    $$
    r = || \begin{pmatrix}
        x \\ y \\ z
    \end{pmatrix} - \begin{pmatrix}
        a \\ b \\ c
    \end{pmatrix} || = \sqrt{(x-a)^2 + (y-b)^2 + (z-c)^2}
    $$

    This gives the equation of a \textbf{sphere}:

    \vspace{-7.5pt}

    $$(x-a)^2 + (y-b)^2 + (z-c)^2 = r^2$$
\end{minipage}};
\node[rounded-box-title, left=10pt] at (box.north east) {Theorem};
\end{tikzpicture}

\switchcolumn

\begin{tikzpicture}
\node [rounded-box] (box){\begin{minipage}{0.45\textwidth}
    \begin{center}
        object dimension + number of general form equations
        
        = dimension of the space
    \end{center}
\end{minipage}};
\node[rounded-box-title, left=10pt] at (box.north east) {The Balancing Formula};
\end{tikzpicture}

\end{paracol}

\vspace{-10pt}

\begin{center}
\begin{tabular}{c|c|c|c|c}
    & Normal Form & General Form & Vector Form & Parametric Form \\
    \hline
    Lines in $\mathbb{R}^2$ & $\mathbf{n} \cdot ( \mathbf{x} - \mathbf{x} ) = \mathbf{0} \iff \mathbf{n} \cdot \mathbf{x} = \mathbf{n} \cdot \mathbf{p}$ & $n_1 x + n_2 y = c$  & $\mathbf{x} = \mathbf{p} + t \mathbf{v}$ & $\begin{cases}
        x = p_1 + t v1 \\
        y = p_2 + t v2
    \end{cases}$ \\
    \hline
    Lines in $\mathbb{R}^3$ & $\begin{cases}
        \mathbf{n}_1 \cdot \mathbf{x} = \mathbf{n}_1 \cdot \mathbf{p}_1 \\
        \mathbf{n}_2 \cdot \mathbf{x} = \mathbf{n}_2 \cdot \mathbf{p}_2
    \end{cases}$ & $\begin{cases}
        n_1{_1} x + n_1{_2} y + n_1{_3} z = c_1 \\
        n_2{_1} x + n_2{_2} y + n_2{_3} z = c_2
    \end{cases}$ & $\mathbf{x} = \mathbf{p} + t \mathbf{v}$ & $\begin{cases}
        x = p_1 + t v1 \\
        y = p_2 + t v2 \\
        z = p_3 + t v3
    \end{cases}$ \\
    \hline
    Planes in $\mathbb{R}^3$ & $\mathbf{n} \cdot ( \mathbf{x} - \mathbf{p} ) = \mathbf{0} \iff \mathbf{n} \cdot \mathbf{x} = \mathbf{n} \cdot \mathbf{p}$ & $n_1 x + n_2 y + n_3 z = c$  & $\mathbf{x} = \mathbf{p} + s \mathbf{u} + t \mathbf{v}$ & $\begin{cases}
        x = p_1 + s u_1 + t v_1 \\
        y = p_2 + s u_2 + t v_2 \\
        z = p_3 + s u_3 + t v_3
    \end{cases}$
\end{tabular}
\end{center}

$\mathbf{n} \neq \mathbf{0}$ is a normal vector. $\mathbf{p}$ is a given point on the line / in the plane. $\mathbf{u}, \mathbf{v} \neq \mathbf{0}$ are direction vectors of the line / plane.

\begin{paracol}{2}

\begin{tikzpicture}
\node [rounded-box] (box){\begin{minipage}{0.45\textwidth}
    The distance from a point $B$ to a line $l$ with a given point $A$ (/ equation $ax + by + \dots = d$) is given by

    \vspace{-15pt}

    $$d(B, l) = || \vec{AB} - \text{Proj}_\mathbf{d}(\vec{AB}) || = \frac{| a x_0 + b y_0 + \dots - d |}{\sqrt{a^2 + b^2 + \dots}}$$
\end{minipage}};
\node[rounded-box-title, left=10pt] at (box.north east) {Theorem};
\end{tikzpicture}

\switchcolumn

\begin{tikzpicture}
\node [rounded-box] (box){\begin{minipage}{0.45\textwidth}
    The distance from a point $B$ to a plane $\mathcal{P}$ with a given point $A$ and normal vector $\mathbf{n}$ is given by

    \vspace{-15pt}

    $$d(B, \mathcal{P}) = || \text{Proj}_\mathbf{n}(\vec{AB}) || = \frac{| a x_0 + b y_0 + c z_0 - d |}{\sqrt{a^2 + b^2 + c^2}}$$
\end{minipage}};
\node[rounded-box-title, left=10pt] at (box.north east) {Theorem};
\end{tikzpicture}

\end{paracol}
