\subsection{Linear Independence}

\begin{paracol}{2}

\begin{tikzpicture}
\node [rounded-box] (box){\begin{minipage}{0.45\textwidth}
    Given vectors $\mathbf{v}_1, \dots, \mathbf{v}_k \in \mathbb{R}^n$, a vector $\mathbf{w}$ is a \textbf{linear combination} of $\mathbf{v}_1, \dots, \mathbf{v}_k$ if, for some non-zero scalars $c_1, \dots, c_k$:
    $$\mathbf{w} = c_1 \mathbf{v}_1 + \dots + c_k \mathbf{v}_k$$
\end{minipage}};
\node[rounded-box-title, left=10pt] at (box.north east) {Definition};
\end{tikzpicture}

\begin{tikzpicture}
\node [rounded-box] (box){\begin{minipage}{0.45\textwidth}
    The set of all possible linear combinations of these vectors is called their \textbf{span}. \\

    That is, if vectors $\mathbf{v}_1, \dots, \mathbf{v}_k$ span $\mathbb{R}^m$, every vector in $\mathbb{R}^m$ can be written as a linear combination of $\mathbf{v}_1, \dots, \mathbf{v}_k$.
\end{minipage}};
\node[rounded-box-title, left=10pt] at (box.north east) {Definition};
\end{tikzpicture}

\textbf{Example}: $\text{Span}(\{\mathbf{u}, \dots, \mathbf{v}\})$ is the plane through $\mathbf{u}, \dots, \mathbf{v}$. \\

A set of vectors is \textbf{linearly dependent} if at least one of the vectors can be written as a linear combination of the others.

\begin{tikzpicture}
\node [rounded-box] (box){\begin{minipage}{0.45\textwidth}
    The \textbf{dependence relation}:
    A set of vectors $\{\mathbf{v}_1, \dots, \mathbf{v}_k\}$ is linearly dependent if there are coefficients $c_1, \dots, c_k \in \mathbb{R}$ such that
    $$c_1 \mathbf{v}_1 + \dots + c_k \mathbf{v}_k = \mathbf{0}$$
\end{minipage}};
\node[rounded-box-title, left=10pt] at (box.north east) {Definition};
\end{tikzpicture}

\begin{tikzpicture}
\node [rounded-box] (box){\begin{minipage}{0.45\textwidth}
    A homogeneous system always has a solution where all the variables are $0$. This is the \textbf{trivial solution}. \\

    A \textbf{non-trivial solution} is a solution where at least one variable is not $0$.
\end{minipage}};
\node[rounded-box-title, left=10pt] at (box.north east) {Definition};
\end{tikzpicture}

\begin{tikzpicture}
\node [rounded-box] (box){\begin{minipage}{0.45\textwidth}
    The following two statements are equivalent:
    \begin{enumerate}
        \item $A\mathbf{x} = \mathbf{0}$ has only the trivial solution.
        \item The set of vectors made up of the columns of $A$, $\{\mathbf{a}_1, \dots, \mathbf{a}_k\}$ is linearly independent.
    \end{enumerate}
\end{minipage}};
\node[rounded-box-title, left=10pt] at (box.north east) {Theorem};
\end{tikzpicture}

\switchcolumn

\begin{tikzpicture}
\node [rounded-box] (box){\begin{minipage}{0.45\textwidth}
    The column vectors of a $m \times n$-matrix $A$ are linearly dependent if and only if the homogeneous linear system $A \mathbf{x} = \mathbf{0}$ has a non-trivial solution.
\end{minipage}};
\node[rounded-box-title, left=10pt] at (box.north east) {Theorem};
\end{tikzpicture}

\begin{tikzpicture}
\node [rounded-box] (box){\begin{minipage}{0.45\textwidth}
    The columns of a matrix $A$ are linearly independent if and only if $\det(A) \neq 0$.
\end{minipage}};
\node[rounded-box-title, left=10pt] at (box.north east) {Theorem};
\end{tikzpicture}

Elementary row operations have a specific effect on the determinant of the matrix:

\begin{itemize}
    \item Swapping rows: This operation flips the sign of the determinant.
    \item Multiplying a row by a non-zero scalar: This operation multiplies the determinant by the same scalar.
    \item Adding a multiple of one row to another: This operation doesn't change the determinant itself.
\end{itemize}

If $\det(A) = 0$, then there exists a non-trivial solution to the system of linear equations represented by the rows of $A$.

\begin{tikzpicture}
\node [rounded-box] (box){\begin{minipage}{0.45\textwidth}
    The row vectors of a $m \times n$-matrix $A$ are linearly dependent if and only if $\text{rank}(A) < m$.
\end{minipage}};
\node[rounded-box-title, left=10pt] at (box.north east) {Theorem};
\end{tikzpicture}

Thus, the rows of a matrix will be linearly dependent if elementary row operations can be used to crate a zero row.

\begin{tikzpicture}
\node [rounded-box] (box){\begin{minipage}{0.45\textwidth}
    Any set of $m$ vectors in $\mathbb{R}^n$ is linearly dependent if $m > n$, i.e. if there are "too many" vectors to be independent.
    \end{minipage}};
\node[rounded-box-title, left=10pt] at (box.north east) {Theorem};
\end{tikzpicture}

\begin{tikzpicture}
\node [rounded-box] (box){\begin{minipage}{0.45\textwidth}
    Any set of vectors containing the zero vector is linearly dependent.
\end{minipage}};
\node[rounded-box-title, left=10pt] at (box.north east) {Theorem};
\end{tikzpicture}

\end{paracol}
